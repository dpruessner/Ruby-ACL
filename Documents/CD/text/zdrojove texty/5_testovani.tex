\chapter{Testování}

Testování bylo zaměřeno pouze na funkcionalitu. Vynecháno bylo testování rychlosti, práce s operační pamětí a spotřeby procesorového času. Testoval jsem jednotkovými testy za pomoci Ruby modulu \verb|Test::Unit| a informací z knihy Ruby - kompendium znalostí \cite{Ruby} a online tutoriálu \cite{ibm:unittesting}.

Jako codecoverage software byl použit SimpleCov rubygem, ale výsledky byly nereprezentativní, nejspíše kvůli špatnému použití. Bohužel jsem se nikde v dokumentaci SimpleCov ani na webu nedopátral správného použití. Codecoverage pokrýval pouze řádky definující metody a třídy a to i v případě, kdy byly všechny testy zakomentovány. Z tohoto faktu usuzuji, že jsem buď nedodržel normu SimpleCov, nebo ho špatně použil. Nicméně si myslím, že pokrytí testy je dobré.

\section{RubyACL}
Testoval jsem každou veřejnou metodu u RubyACL. Celkem jsem vytvořil 43 testů s 119 porovnáními (anglicky assertion). Testy jsem rozdělil do 4 částí-souborů podle charakteru metod, které testuji. Set1 testuje metody přímo spjaté s ACL, například load, save and rename. Set2 testuje metody, které vytváří ACL objekty. Set3 testuje metody, které upravují ACL objekty. Set4 testuje pouze metodu check.
Doba běhu testovací soupravy (anglicky test suit) trvá delší dobu než bývá obvyklé, protože před každým spuštěním nového testu se do databáze nahrávájí testovací dokumenty a po každém testu se maže kolekce s testovacími daty. Hlavně nahrávání dokumentů probíhá poměrně dlouhou dobu. Čas provedení testů je přibližně 80 sekund.

\section{eXistAPI}
Testoval jsem každou veřejnou metodu eXistAPI. Celkem jsem vytvořil 14 testů s 21 porovnáními (anglicky assertion). 