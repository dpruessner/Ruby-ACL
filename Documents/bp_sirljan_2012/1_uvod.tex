\chapter{Úvod}

\section{Úvod do problematiky}

Tato bakalářská práce navazuje na můj semestrální projekt. Zabývá se správou - administrací řízení přístupu, jakožto procesu autorizování uživatelů, skupin a počítačů pro přístup k objektům. Tento proces pracuje s pojmy: oprávnění, uživatelská práva a audit objektů. Tato jednotlivá přiřazená oprávnění začleňuje jako položky řízení přístupu (ACE\footnote{Access Control Entry}) a jejich celé sady začleňuje do seznamu přístupových práv (ACL\footnote{Access Control List}). Úkolem bakalářské práce bylo vytvořit, navrhnout a realizovat v jazyce Ruby model uživatelských přístupových práv určený pro objektovou XML\footnote{Extensible Markup Language} databázi.

Součástí práce byl návrh knihovny. Navržená knihovna realizuje správu řízení přístupu pomocí ACL. Knihovnu nazývám Ruby-ACL. Protože moje bakalářská práce je prací implementační, včetně testů navržené knihovny, zaměřil jsem se na specifikaci rozhraní knihovny a na příklady jejího použití. Výsledkem je nejen samotná realizace knihovny, ale i podrobná programátorská dokumentace.

Je-li potřeba zabezpečit zdroje a jeho prostředky, je nutné vzít v úvahu, jakými právy budou disponovat ti, kteří k nim budou přistupovat. Zabezpečit objekt, například dokument či kolekci, lze přidělením oprávnění, která umožňují uživatelům nebo skupinám provádět u tohoto objektu určité akce. Řízení přístupů je činnost zabývající se přidáváním a zamítáním oprávnění přistupujícím.

Ruby je objektově orientovaný, interpretovaný skriptovací programovací jazyk. Díky své jednoduché syntaxi je poměrně snadný, přesto však dostatečně výkonný, aby dokázal konkurovat známějším jazykům jako je Python a Perl. Převzato a upraveno z wikipedie\cite{wiki:Ruby}

ACL je seznam pro řízení přístupů. Seznam určuje, kdo nebo co má povolení přistupovat k objektu a jaké operace s ním může nebo nesmí provádět. V typickém ACL specifikuje každý záznam v seznamu uživatele a operaci\cite{wiki:acl}.

%-------------------------------

\section{Motivace}
Zaujala mě problematika práv, databází a pro mě neznámého jazyka Ruby.
Jádro celé mé motivace, bylo projít si vývojem softwaru, v tomto případě knihovny, od návrhu přes implementaci k testování a dokumentaci a ověřit si získané poznatky z předmětu softwarového inženýrství. Soustředil jsem se na vlastní nápady. Nechtěl jsem skládat a kopírovat polotvary a "lepit" je dohromady.