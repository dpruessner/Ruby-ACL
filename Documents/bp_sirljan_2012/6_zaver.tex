\chapter{Závěr}

\begin{itemize}
\item Zhodnocení splnění cílů DP/BP a  vlastního přínosu práce (při formulaci je třeba vzít v potaz zadání práce).
\item Diskuse dalšího možného pokračování práce.
\end{itemize} 

Časově nejnáročnější operací je komunikace s databází. Zrychlení by se dalo docílit optimalizací dotazování tak, aby se dotazovalo, co nejméně. Záporem tohoto řešení by byla větší paměťová náročnost aplikace, která by knihovnu Ruby-ACL používala. Současný kód knihovny obsahuje opakované dotazování na stejnou věc místo ukládání výsledku do paměti pro případné následující použití. Tento model byl zvolen kvůli jednoduchosti a faktu, že vykonání dotazu a vrácení výsledku trvá řadově milisekundy. Je ale těžké odhadnout, kolik času by zabralo dotazování při plné databázi a použití více uživatelů ve stejný čas. Navíc eXist-db si po určitou dobu v paměti uchovává výsledky předchozích dotazů, čímž se ještě více snižuje naročnost opakovaných dotazů. Vše je ale závislé na lince mezi knihovnou a databází.