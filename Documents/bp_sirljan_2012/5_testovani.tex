\chapter{Testování}

\begin{itemize}
 \item Způsob, průběh a výsledky testování.
 \item Srovnání s existujícími řešeními, pokud jsou známy.
\end{itemize} 


Testování bylo zaměřeno pouza na funkcionalitu. Vynecháno bylo testování rychlosti, práce s operační pamětí, spotřeby procesorového času. Testoval jsem unit testama za pomocí Ruby modulu \verb|Test::Unit| a informací z knihy Ruby - kompendium znalostí \cite{Ruby} a online tutoriálu \cite{ibm:unittesting}.
Test Coverage TODO

\section{RubyACL}
Testoval jsem každou veřejnou metodu u RubyACL. Celkem jsem vytvořil 43 testů s 119 asercema. Testy jsem rozdělil do 4 částí-souborů podle charakteru metod, které testují. Set1 testuje metody přímo spjaté s ACL, například load, save and rename. Set2 testuje metody, které vytváří ACL objekty. Set3 testuje metody, které upravují ACL objekty. Set4 testuje pouze metodu check.
Testování probíhá neobvykle delší dobu, protože před každým spuštěním nového testu se do databáze nahrávájí testovací dokumenty a po každém testu se maže kolekce s testovacími daty. Hlavně nahrávání dokumentů probíhá poměrně dlouhou dobu. Čas provedení testů trvá asi 80 sekund.

\section{eXistAPI}
Testoval jsem každou veřejnou metodu eXistAPI. Celkem jsem vytvořil 14 testů s 21 asercema. 