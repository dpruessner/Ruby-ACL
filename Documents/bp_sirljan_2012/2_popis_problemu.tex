\chapter{Popis problému, specifikace cíle}

\section{Popis řešeného problému}

Databáze, pro kterou byla knihovna určena, nemá žádný model uživatelských přístupových práv. Bylo potřeba tento nedostatek vyřešit knihovnou implementovanou v jazyce Ruby. Jazyk Ruby byl vybrán z důvodu jeho rozšířenosti a z důvodu kompatibilty s budoucími částmi databáze, které budou taktéž naimplementované v Ruby. Mnou naimplementovaná knihovna Ruby-ACL řeší problém se spravováním přístupových práv.

\section{Vymezení cílů a požadavků}
Cílem bakalářská práce bylo navrhnout, realizovat a otestovat knihovnu v jazyce Ruby, která bude spravovat uživatelská přístupová práva pro objektovou XML databázi.
Cílem bylo vytvořit co nejjednodušší knihovnu, která by splňovala všechna kritéria zadání. Nechtěl jsem používat nebo skládat dohromady existující moduly a knihovny, které danou problematiku řeší, protože jsem si za cíl dal vytvořit něco svého a projít si vývojem softwaru od požadavků, analýzy, návrhu přes realizaci k testování a dokumentaci. 
I když Ruby-ACL je určena pro XML databázi, chtěl jsem, aby byla použitelná i pro reálné přístupy do budov apod.
Předsevzal jsem si, že by bylo přínosné, kdyby knihovna umožňovala jemně nastavit přístupy (v angličtině se používá výraz "fine-grained").

\noindent
Seznam požadavků je popsán v tabulce  \ref{tab:tab1}

\begin{table}%[t]
\begin{center}
\begin{tabular}{|l|p{9cm}|l|}
\hline
\textbf{id} & \textbf{Specifikace požadavků na software} & \textbf{priorita} \\
\hline
\multicolumn{3}{|l|}{FUNKČNÍ POŽADAVKY} \\
\hline
0 & Ruby-ACL bude umožňovat řízení přístupů pomocí ACL & povinný\\
\hline
1.0 & Ruby-ACL bude umožňovat definovat, editovat a mazat zmocnitele (principals) & povinný\\
\hline
1.1 & Ruby-ACL bude umožňovat definovat, editovat a mazat oprávnění (privileges) & povinný\\
\hline
1.2 & Ruby-ACL bude umožňovat definovat, editovat a mazat  zdrojové objekty (resource objects) & povinný\\
\hline
1.3 & Ruby-ACL bude umožňovat definovat, editovat a mazat pravidla přístupu (ACE) & povinný\\
\hline
2.0 & Ruby-ACL bude umožňovat vytvářet ACL & povinný\\
\hline
2.1 & Ruby-ACL bude umožňovat načítat a ukládat ACL z a do XML souboru & povinný\\
\hline
2.2 & XML soubor bude definovaný pomocí DTD or XML Schema & volitelný\\
\hline
2.3 & XML soubor bude "well formated" podle W3C doporučení & povinný\\
\hline
3.0 & Ruby-ACL bude nabízet pouze Default-Deny politiku & povinný\\
\hline
4.0 & Ruby-ACL bude testována na eXist-db & povinný\\
\hline
\multicolumn{3}{|l|}{OBECNÉ POŽADAVKY} \\
\hline
1.0 & Ruby-ACL bude naprogramována v jazyce Ruby & povinný \\
\hline
2.0 & Ruby-ACL bude vydaná jako RubyGem & volitelný\\
\hline
3.0 & Ruby-ACL potřebuje databázi podporující xQuery, xPath technologie & povinný\\
\hline
\end{tabular}
\end{center}
\caption{Tabulka funkčních a obecných požadavků. priorita = (povinný, volitelný, nepovinný)}
\label{tab:tab1}
\end{table}

\section{Popis struktury bakalářské práce}

Nejpodstatnější částí bakalářské práce z pohledu vytyčených cílů je sekce Rozhraní a sekce Příklady užití v kapitole Analýza a dále celá kapitola Testování.

Kapitola 1 nás uvádí do Bakalářské práce. Vysvětluje, co vlastně řízení práv je, popisuje jeho význam a vysvětluje nejdůležitější pojmy.

V kapitole 2 se hovoří o důvodech implementace Ruby-ACL, vytyčují se cíle a prezentují požadavky. Dále obsahuje stručný popis existujících řešeních.

Kapitola 3 je nejpodstatnější z celé práce. Jedná se o Analýzu, ve které bylo popsáno z čeho jsem vycházel při návrhu. Nejdůležitější část je Rozhraní a Ukázka použítí, protože přímo splňují zadání/požadavky práce. Sekce Popisu logiky vyhodnocování pravidel je důležitá pro uživatele knihovny. Nejen že vysvětluje pojmy jako ACL objekt, principal, ale hlavně popisuje jakým způsobem knihovna Ruby-ACL rozhoduje, které pravidlo má vyšší váhu.

Kapitola 4 se zaměřuje na postup vývoje při implementaci a problémy při realizaci.

Kapitole 5 je rozdělená na dvě části. Jedna se zabýva pomocným komunikačním rozhraním eXistAPI a druhá samotnou knihovnou Ruby-ACL. Obě části jsou však zaměřeny na vysvětlení způsobu testování a jejich výsledky.

Kapitola 6 se zaměřuje na zhodnocení splnění cílů Bakalářské práce a rozebírá možné nedostatky a případné pokračování v práci na knihovně.

V příloze se nacházejí diagramy, které nebyly potřeba pro vysvětlení funkcionality knihovny. Součastí přílohy je i postup instalace knihovny a CD s knihovnou a dalšími soubory.



\section{Existující řešení}
Existující řešení lze rozdělil na dva druhy. První jsou firemní řešení a druhé jsou knihovny v Ruby zabývající se stejnou nebo podobnou problematikou.

\subsection{Firemní řešení}
Vybral jsem tři ukázková řešení.
Oracle má nejpropracovanější model řízení práv. Podporuje integrování LDAP\footnote{Lightweight Directory Access Protocol} a WebDAV\footnote{Web-based Distributed Authoring and Versioning}.

PhpGACL je bezplatný jednodušší systém ve srovnání Oracle spravující přístupy ke zdrojům. PhpGACL je opensource využívaný pro webové aplikace.

Obecné řešení se skládá z dvojrozměrné tabulky, kde jeden rozměr je tvořen všemi, kdo vyžadují přístup a druhý rozměr obsahuje objekty, ke kterým je vyžadován přístup. Oprávnění je v buňce, která se nachází na prusečíku os zminěných dvou rozměrů.
Podrobnější zpracování se nachází v sekci \ref{sec:anal-existujicireseni} Existující řešení, která se nachází v kapitole Analýza.

\subsection{Dostupné knihovny}
V následujících podsekcích jsou dostupné knihovny napsané v Ruby, které by mohly být použity, kdybych si za cíl nedal vytvořit Ruby-ACL sám.

\subsubsection{Acl9}
Acl9 je další řešení autorizace založené na rolích v Ruby on Rails\footnote{framework pro pohodlné a rychlé vytváření moderních webových aplikací}. Skládá se ze dvou subsystémů, které mohou být použity samostatně. Subsystém kontroly rolí umožňuje nastavovat a dotazovat se na uživatelské role pro různé objekty. 
Subsystém řízení přístupu umožňuje určit různá přístupová pravidla založená na rolích uvnitř řadičů. Text byl převzat a upraven z \cite{github:acl9}

\subsubsection{iq-acl}
Cílem tohoto rubygemu je poskytnout serii tříd, které umí zacházet s bežnými požadavky na řízení práv. V současné době poskytuje třídu \verb|IQ::ACL::Basic|, která přestože je velmi jednoduchá je také velmi schopná.Více o použití se můžete dočíst zde \cite{github:iq-acl}.

\subsubsection{ActiveACLPlus}
Plugin ActiveAclPlus realizuje flexibilní, rychlý a snadno použitelný obecný systém kontroly přístupu.
Systém je založen na phpgacl.sourceforge.net, přidáním objektu orientace, polymorfismu a dvou úrovní paměti. PhpGacl tvrdí, že v reálné pracovní verzi s mnoha přidanými vrstvami složitosti podporuje více než 60.000 účtů, 200 skupin a 300 ACO." Testy provedené na vyvojářském notebooku ukazují 10 - 30 krát větší zlepšení výkonnosti ve srovnání s \verb|active_rbac|.
Plugin používá ukládání do mezipaměti. Používá instanční pamět a v případě potřeby ukládá výsledky oprávnění do paměti s použitím časového omezení. Text byl převzat a upraven z \cite{rubyforge:ActiveAclPlus}